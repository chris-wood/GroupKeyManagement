%
% Hello! Here's how this works:
%
% You edit the source code here on the left, and the preview on the
% right shows you the result within a few seconds.
%
% Bookmark this page and share the URL with your co-authors. They can
% edit at the same time!
%
% You can upload figures, bibliographies, custom classes and
% styles using the files menu.
%
% If you're new to LaTeX, the wikibook at
% http://en.wikibooks.org/wiki/LaTeX
% is a great place to start, and there are some examples in this
% document, too.
%
% Enjoy!
%
\documentclass[12pt]{article}

\usepackage[english]{babel}
\usepackage[utf8x]{inputenc}
\usepackage{amsmath}
\usepackage{graphicx}
\usepackage{algorithm2e}

\title{Verification of the Statistical Model for Multi-Stage Message Distribution}
\author{Jedi High Council}

\begin{document}
\maketitle

\section{Monte Carlo Simulation}

As in the first paper, we use a monte carlo simulation to verify the correctness of the statistical model developed for multi-stage message distribution. A high-level overview of the simulation algorithm is detailed in Algorithm \ref{alg:simulation}. Both the Matlab source code for the simulation and Java source code for the model should be linked to in the final paper.

Due to the computational complexity of the model, the current unoptimized code can not go beyond $n = 9$ nodes and $n, m >= 3$. The recursive process of generating candidate transition matrices from a particular $D^a$ subspace to $D^{a+1}$ is very expensive. I think it can be optimized by a simple pruning scheme to remove redundant branches of the search space, though this has yet to be explored. 

A list of the times from the model and simulation for $T = 10000$ runs, along with the simulation standard deviation and standard error, is shown in the attached spreadsheet. The model and simulation appear to match. 

\begin{algorithm}[H] \label{alg:simulation}
  \SetAlgoLined
  \KwData{$T$, $k$, $m$, $n$, $p_1$, and $p_2$}
  \KwResult{Expected time}
  	$A_c \gets zeros[1\dots n][1\dots n]$\;
    $A_m \gets zeros[1\dots k][1\dots n][1\dots n]$\;
    $n_c \gets 0$\;
    $C_l \gets zeros[1\dots n]$\;
    $total \gets 0$\;
	\For{$T_i = 0 \to T$} {
    	$t \gets 0$\;
    	\While{$n_c < n - 1$} {
        	Build a list of candidate child nodes ready to receive a new message (i.e. those unconnected and not receiving a message already). Filter the list by randomly discarding each candidate node with probability $1 - p_1$\
            \For{$m_i = 0 \to m$} {
            	With probability $p_2$, advance each child node in stage $S_{m_i}$ to $S_{m_{i+1}}$. If a node advances to stage $S_m$, set them as connected in $C_l$, update their connection with the parent in $A_c$, and discard their message trace in $A_m$.\;
        	}
            Randomly assign each child node in the message ready list to an available parent. If the number of available parents is less than the number of ready children, then a subset of those in the ready list begin a communication trace. Else, every children begins a communication trace.\;
            $t \gets t + 1$\;
        }
        $total \gets total + t$\;
    }
    \textbf{output} $total / T$
  \caption{Monte carlo simulation to verify the statistical model}
\end{algorithm}


\end{document}