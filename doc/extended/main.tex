%
% Hello! Here's how this works:
%
% You edit the source code here on the left, and the preview on the
% right shows you the result within a few seconds.
%
% Bookmark this page and share the URL with your co-authors. They can
% edit at the same time!
%
% You can upload figures, bibliographies, custom classes and
% styles using the files menu.
%
% If you're new to LaTeX, the wikibook at
% http://en.wikibooks.org/wiki/LaTeX
% is a great place to start, and there are some examples in this
% document, too.
%
% Enjoy!
%
\documentclass[12pt]{article}

\usepackage[english]{babel}
\usepackage[utf8x]{inputenc}
\usepackage{amsmath}
\usepackage{graphicx}
\usepackage{algorithm2e}

\title{Verification of the Statistical Model for Multi-Stage Message Distribution}
\author{Jedi High Council}

\begin{document}
\maketitle

\section{Monte Carlo Simulation}

To verify our model we implemented a monte carlo simulation that simulates the behavior of the network given parameters $n$, $m$, and $k$. In particular, the simulation initializes the network of $n$ nodes in an unconnected state (i.e. all nodes are disconnected from each other and there exists a single node, the key manager, who possesses the key). Then, in discrete time steps, the simulation attempts to establish new connections between a pair of nodes $u$ and $v$ with probability $p_1$. In addition, nodes already in the process of establishing a connection proceed forward through the $m$ stages of message distribution by a single step with probability $p_2$. At the end of one time step, the overall discrete time is incremented by one, and the process repeats until all nodes have been connected and received the key. This procedure is formalized in Algorithm \ref{alg:simulation}. The source code, written in Matlab, has been made available at <LINK> for interested readers to experiment with. It is a self contained program that contains all of the required documentation within.

We repeated this general procedure for $T = 10000$ iterations each for different values of $n$, $k$, and $m$, collecting the average time over all program runs, standard deviation of each run, and the estimated error in the simulation. Our estimated error was quite small and well within the expected bounds of precision. We then computed the difference $\Delta$ between the output from our model and the average time reported by this monte carlo simulation to verify the correctness. We found that for all configurations considered the delta was always less than $0.2$. We attribute this difference to the error introduced in monte carlo simulation. \\

\begin{algorithm}[H] \label{alg:simulation}
\tiny
  \SetAlgoLined
  \KwData{$T$, $k$, $m$, $n$, $p_1$, and $p_2$}
  \KwResult{Expected time}
    $total \gets 0$\;
	\For{$T_i = 0$ to $T$} {
    	$t \gets 0$\;
      $A_c \gets zeros[1\dots n][1\dots n]$\;
      $A_m \gets zeros[1\dots k][1\dots n][1\dots n]$\;
      $n_c \gets 0$\;
      $C_l \gets zeros[1\dots n]$\;
    	\While{$n_c < n - 1$} {
        	$L \gets getReadyChildren(A_c, A_m, C_l, 1-p_1)\;$\\
          %Build a list of candidate child nodes ready to receive a new message (i.e. those unconnected and not receiving a message already). \\
          %Filter the list by randomly discarding each candidate node with probability $1 - p_1$\\
          \For{$r = 1$ to $n$} {
            \For{$c = 1$ to $n$} {
              \If{$A_m[r][c][m] = 1$} {
                $A_m[r][c][m] \gets 0\;$ \\
                $A_c[r][c] \gets 1\;$ \\
                $A_c[c][r] \gets 1\;$ \\
                $C_l[c] \gets 1\;$ \\
                $n_c \gets n_c + 1\;$\\
              }
            }
          }
          \For{$m' = 1$ to $m - 1$} {
            \For{$r = 1$ to $n$} {
              \For{$c = 1$ to $n$} {
                \If{$A_m[r][c][m'] = 1$ and $rand() < p_2$} {
                  $A_m[r][c][m' + 1] \gets 1\;$ \\
                  $A_m[r][c][m'] \gets 0\;$ \\
                }
              }
            }
          }
          $P \gets getReadyParents(A_c, A_m, A_l)\;$\\
          \For{$i \gets 0$ to $min\{|P|, |L|\}$} {
            $A_m[P[i]][L[i]][1] \gets 1\;$ \\
          }
          $t \gets t + 1$\;
        }
         %    	With probability $p_2$, advance each child node in stage $S_{m_i}$ to $S_{m_{i+1}}$. If a node advances to stage $S_m$, set them as connected in $C_l$, update their connection with the parent in $A_c$, and discard their message trace in $A_m$.\;
        	% }
         %    Randomly assign each child node in the message ready list to an available parent. If the number of available parents is less than the number of ready children, then a subset of those in the ready list begin a communication trace. Else, every children begins a communication trace.\;
         $total \gets total + (t - 1)$\;
      }
    \textbf{output} $total / T$
  \caption{Monte carlo simulation to verify the statistical model. The functions $getReadyChildren()$ and $getReadyParents()$ uniformly select nodes from the entire group at random to be new children and parents, respectively, by analyzing the current state of the network as represented by the adjacency matrix $A_m$, message matrix $A_m$, and connected list $C_l$. Also, $getReadyChildren()$ uses probability $1- p_1$ when selecting new children to establish connections with.}
\end{algorithm}

\end{document}